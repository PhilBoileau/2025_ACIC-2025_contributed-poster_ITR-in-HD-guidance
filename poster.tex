\documentclass[25pt, a1paper, landscape, innermargin=-3in]{tikzposter}

% required packages
\usepackage{amsmath}
\usepackage{amsfonts}
\usepackage{amsthm}
\usepackage{graphicx}
\usepackage[square,numbers]{natbib}
\usepackage{url}
\usepackage{booktabs}
\usepackage{geometry}

% set the appropriate paper size
\geometry{papersize={40in,42in}}

% compact bibliography
\renewcommand{\bibsection}{}
\setlength{\bibsep}{0pt plus 0.3ex}

%% Available themes: see also
%% https://bitbucket.org/surmann/tikzposter/downloads/themes.pdf

\definecolorpalette{sampleColorPalette} {
  \definecolor{colorOne}{named}{green}
  \definecolor{colorTwo}{named}{black}
  \definecolor{colorThree}{named}{cyan}
}
\usetheme{Default}
%% \usecolorstyle{Denmark}
\colorlet{backgroundcolor}{white}
\colorlet{framecolor}{gray}
\colorlet{blocktitlebgcolor}{gray}

% title information
\title{Guidance on Individualized Treatment Rule Estimation in High Dimensions}
\author{Philippe Boileau$^1$, Ning Leng$^2$, Sandrine Dudoit$^3$}
\institute{$^1$McGill University; $^2$Genentech, Inc.; $^3$University of California, Berkeley}

% dictate default block options
\newcommand{\myblock}[2]{\block[titleinnersep=2mm, linewidth=0.5mm]{#1}{#2}}
\newcommand{\mysmallblock}[2]{\block[titleinnersep=1mm, linewidth=1mm, bodyinnersep=5mm, roundedcorners=12, ]{{\small #1}}{{\scriptsize#2\par}}}

% turn off
\tikzposterlatexaffectionproofoff
\begin{document}
\maketitle[width=34in]


\begin{columns}

  \column{0.333}

  \myblock{Motivation}{

    Precision medicine promises to tailor patients' treatments to optimize their
    outcomes. Patients' pre-treatment covariates, like age, sex-at-birth, and
    genetic profiles, influence therapies' efficacy, safety, and tolerability.
    These patient characteristics, known as \textbf{treatment effect modifiers,
      are used to derive individualized treatment rules (ITRs) that guide
      personalized treatment decisions}.

    \vspace{0.1in}

    While numerous methods can successfully estimate ITRs in traditional
    asymptotic settings, \textbf{learning ITRs from high-dimensional data with
      more pre-treatment covariates than patients --- a common occurrence in
      modern clinical data --- is challenging}. ITR estimators developed for
    such data rely on simplifying assumptions. Recent advances in
      treatment effect modifier detection may help.

    \vspace{0.1in}

    A comparison of ITR estimation procedures in high-dimensional settings has
    not been performed. Nor has an evaluation of these methods' sensitivities to
    assumption violations. As such, \textbf{selecting an appropriate ITR
      estimation strategy for high-dimensional settings is challenging for
      applied biomedical researchers}. Capacity for precision medicine is
    diminished as a result.

  }

  \myblock{Primary Objectives}{
    \begin{itemize}
        \item \textbf{Provide guidance based on practical operating
          characteristics to applied scientists for ITR estimation in high
          dimensions.}
        \item \textbf{Determine whether treatment effect modifier detection
          procedures improve ITR estimation in high dimensions.}
    \end{itemize}
  }

  \myblock{Operating Characteristics}{
    \begin{itemize}
        \item \textbf{Rule Quality:} An ITR is ``high-quality'' when its
          expected outcome approaches that of the optimal rule. That is, the
          rule approximately optimizes mean outcome in the population.

        \item \textbf{Accurate Interpretability:} An ITR is accurately
          interpretable when it recovers treatment effect modifiers reliably in
          terms of the false discovery rate (FDR) and true positive rate (TPR).

        \item \textbf{Computational Efficiency:} An ITR is computationally
          efficient when it can be estimated quickly in serial with few
          computational resources.

    \end{itemize}
  }

  \myblock{Problem Formulation}{

    Consider $n$ i.i.d.~observations $O = (W, A, Y)$ where $W$ is a $p$-length
    random vector of pre-treatment covariates (and possible confounders) where
    $p \approx n$ or $p > n$, $A$ is a binary treatment indicator, and $Y$ is
    the continuous outcome.

    \vspace{0.1in}

    We aim to estimate the ITR, defined as
    \begin{equation*}
      I(\mathbb{E}[Y|W, 1] - \mathbb{E}[Y|W, 0] > 0) \;,
    \end{equation*}
    where $\mathbb{E}[Y|W, 1] - \mathbb{E}[Y|W, 0]$ is the conditional average
    treatment effect (CATE) under standard identifiability
    conditions.

  }

  \mysmallblock{References}{
    \bibliographystyle{unsrtnat}
    \bibliography{refs}
  }

  \column{0.334}

  \myblock{Estimators}{

    The CATE estimators in the table below were used to construct ITR
    estimators. Filtered versions of these CATE estimators relying on the
    treatment effect modifier variable importance parameter (TEM-VIP)
    methodology of \citet{boileauNonparametricFrameworkTreatment2025} were used
    to construct ITR estimators as well.

    \begin{tikzfigure}
      \centering
        \begin{tabular}{
            |p{3in} || p{8.5in}|
          }
          \hline
          CATE Estimator
          & Details \\
          \hline\hline
          Plug-In LASSO
          & A plug-in estimator using LASSO \citep{tibshiraniRegressionShrinkageSelection1996}. \\
          \hline
          Plug-In XGBoost
          & A plug-in estimator using XGBoost \citep{chenXGBoostScalableTree2016}. \\
          \hline
          Modified Covariates LASSO
          & A modified covariates estimator
          \citep{tianSimpleMethodEstimating2014} using LASSO. The propensity
          score is estimated using logistic LASSO. \\
          \hline
          Modified Covariates XGBoost
          & A modified covariates estimator
          \citep{tianSimpleMethodEstimating2014} using XGBoost. The propensity
          score is estimated using logistic LASSO. \\
          \hline
          Augmented Modified Covariates LASSO
          & An augmented modified covariates estimator
          \citep{tianSimpleMethodEstimating2014} using LASSO. The propensity
          score is estimated using logistic LASSO. \\
          \hline
          Augmented Modified Covariates XGBoost
          & An augmented modified covariates estimator
          \citep{tianSimpleMethodEstimating2014} using XGBoost. The propensity
          score is estimated using logistic LASSO. \\
          \hline
          AIPW-based LASSO
          & An AIPW-based estimator
          \citep{luedtkeSuperLearningOptimalDynamic2016} using Super Learners
          \citep{laanSuperLearner2007} to estimate the
          expected conditional outcome and the propensity score. Differences
          in predicted pseudo-outcomes are modeled using LASSO. \\
          \hline
          AIPW-based Super Learner
          & An AIPW-based estimator
          \citep{luedtkeSuperLearningOptimalDynamic2016} using Super Learners to estimate the
          expected conditional outcome and the propensity score. Differences
          in predicted pseudo-outcomes are modeled using a Super Learner. \\
          \hline
          Causal Random Forests
          & A causal random forest estimator
          \citep{wagerEstimationInferenceHeterogeneous2018} using
          cross-validation for hyperparameter selection. \\
          \hline
      \end{tabular}
    \end{tikzfigure}

  }

  \myblock{Simulated Data-Generating Processes}{

    \textbf{ITR estimators were benchmarked in 16 data-generating processes}
    with continuous outcomes and binary treatment assignments \textbf{reflecting
      a diversity of randomized and observational studies}. Realizations of
    random vector $O$ with $p=500$ were generated according to the following
    data-generating process template:
    \begin{align*}
      W & \sim N(0, \Sigma) \\
      A|W & \sim \text{Bernoulli}(\pi(W)) \\
      Y|W, A & \sim N(\mu(W, A), 1) \;.
    \end{align*}
    Here, $\Sigma$ is some $500 \times 500$ covariance matrix,
    $\pi(W)=\mathbb{P}[A=1|W]$, and $\mu(W,A) = \mathbb{E}[Y|W, A]$.
    Data-generating processes are defined using combinations of the following
    factors:
    \begin{equation*}
      \begin{split}
        \Sigma_{1} & = I_{500 \times 500} \\
        \Sigma_{2} & = \text{Block diagonal} \\
      \end{split}
    \end{equation*}
    \begin{equation*}
      \times
    \end{equation*}
    \begin{equation*}
      \begin{split}
        \pi_{1}(W) & = \frac{1}{2} \\
        \pi_{2}(W) & = \text{logit}^{-1}\left(\frac{W_{1} + W_{2} + W_{3} + W_{4}}{5}\right) \\
      \end{split}
    \end{equation*}
    \begin{equation*}
      \times
    \end{equation*}
    \begin{equation*}
      \begin{split}
        \mu_{1}(A, W) & = A + \gamma^{\top}W + (\delta^{(10)})^{\top}WA \\
        \mu_{2}(A, W) & = A + \gamma^{\top}W + (\delta^{(50)})^{\top}WA \\
        \mu_{3}(A, W) & = \gamma^{\top}W + 2\;\text{arctan}\left\{(\delta^{(10)})^{\top}WA\right\} \\
        \mu_{4}(A, W) & = \gamma^{\top}W + 2\;\text{arctan}\left\{(\delta^{(50)})^{\top}WA\right\} \\
      \end{split}
    \end{equation*}

  }

  \column{0.333}

  \myblock{Results Snapshot}{
    \vspace{-0.5in}

    \begin{tikzfigure}
      \includegraphics[width=0.26\textwidth]{figs/summary-plot.jpeg}
    \end{tikzfigure}

    \vspace{-0.2in}

  }

  \myblock{Practical Guidance}{

    \textbf{No estimator uniformly dominates others across all operating
      characteristics; tradeoffs must be made.} Since rule quality is generally
    of primary concern, practitioners must choose between accurate
    interpretability and computationally efficiency:

    \begin{itemize}

    \item \textbf{High-quality ITRs that are accurately interpretable:} The
      \textbf{TEM-VIP-filtered plug-in LASSO and AIPW-based estimators}
      generally produce high-quality ITR estimates while accurately recovering
      treatment effect modifiers. These estimators are computationally
      intensive, however. The computational burden might be lessened by
      parallelizing the estimation procedure.

    \item \textbf{High-quality ITRs that require few computational resources:}
      The \textbf{plug-in LASSO estimator} produces among the most high-quality
      rules in our simulation studies, providing empirical evidence that it is
      robust to model misspecification while being exceptionally computationally
      efficient. This estimator's built-in feature selection capabilities should
      not be used for treatment effect modifier discovery, however.

    \end{itemize}
  }

  \begin{subcolumns}
    \subcolumn{0.8}

    \mysmallblock{Funding}{

      P.B. gratefully acknowledges the support of the Fonds de recherche du
      Québec - Nature et technologies and the Natural Sciences and Engineering
      Research Council of Canada.

    }
  \end{subcolumns}


\end{columns}

\node [above left, outer sep=2cm] at (bottomleft -| topright) {\includegraphics[width=6cm]{logos/qr-code}};

\end{document}
